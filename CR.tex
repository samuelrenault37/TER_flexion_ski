\documentclass{article}
\usepackage{graphicx} % Required for inserting images

\title{TRAVAIL D'ETUDE ET DE RECHERCHE
ETUDE ET MODELISATION
DE LA FLEXION SUR 
DES SKIS}

\author{Noah RUCHAUD--LANDRAUD\\ Martin POTAGNIK\\Samuel RENAULT\\ Seydou SECK \\\\Tuteurs : Olivier CATY - Marc DURUFLE}


\date{Année Scolaire 2025-2026}


\maketitle

\begin{figure}[h!]
    \centering
    \includegraphics[width=0.32\textwidth]{IMG_2654.png}
    \includegraphics[width=0.32\textwidth]{IMG_2655.png}
    \includegraphics[width=0.32\textwidth]{IMG_2656.png}
    \caption{Récapitulatif de la séance 1 : (a) Ojectifs semestriels, (b) Carte mentale du projet, (c) Avancée travail hebdomadaire}
    \label{fig:flexion_ski}
\end{figure}

\usepackage{graphicx}

\section{\textbf{Idée plan de travail et d'avancée}}

\subsection{Modélisation mécanique du ski comme poutre (Théorie)}

\begin{itemize}
  \item Traitement du ski comme une \textbf{poutre} (linéaire ou non), avec une raideur en flexion $EI(x)$ variable selon la position $x$.
  \item Étude du \textbf{couplage torsion / flexion} : introduction du terme de raideur $GJ$ pour la torsion, pertinent si le ski est soumis à un moment de torsion en plus de la flexion.
  \item Analyse des \textbf{hypothèses de base} : petites déformations, comportement linéaire, homogénéité des matériaux.
  \item Discussion sur les \textbf{limites du modèle} : effets non-linéaires, anisotropie, hétérogénéité des couches.
\end{itemize}
\newpage
\subsection{Mesure expérimentale de la flexion et de la rigidité (Réalité)}

\begin{itemize}
  \item Réalisation d'essais en \textbf{flexion 3 points, 4 points} ou avec appuis multiples pour déterminer le profil de raideur du ski. Idée de base avec la réalisation de l'exercice 1 du TD de RDM au préalable numériquement.
  \item Application de \textbf{charges connues} et mesure de la \textbf{déflexion} correspondante.
  \item Utilisation de \textbf{capteurs} (déplacement, jauges de contrainte, LVDT, capteurs de déformation) placés le long du ski. A voir si le matériel et si des capteurs sont disponibles au CRMI
  \item Comparaison des \textbf{données expérimentales} calculées avec les \textbf{modèles analytiques et numériques} qu'on trouverait en bibliographie sur Internet.
\end{itemize}

\subsection{Effet de la liaison ski-chaussure}

\begin{itemize}
  \item Étude de la manière dont la \textbf{fixation} et la \textbf{rigidité de la chaussure} influencent la distribution des efforts dans le ski.
  \item Analyse du \textbf{transfert des moments de flexion} de la chaussure vers le ski.
  \item Étude de l'\textbf{interaction du système complet} (skieur + chaussure + ski).
  \item   Etude en cas de \textbf{variations du sol} (bosses)

\end{itemize}

\subsection{Comportement en conditions réelles (dynamique)}

\begin{itemize}
  \item Mesure de la \textbf{déflexion du ski en mouvement} pendant les virages, à l’aide de capteurs embarqués (semble beaucoup plus difficile à réaliser expérimentalement).
  \item Étude de l’influence de la \textbf{vitesse}, de l’\textbf{angle d’attaque} et de la \textbf{charge dynamique}(mouvements du skieur, vibration du ski, changement de contact avec la neige).
  \item Analyse de l'importance de la position de la projection du \textbf{centre de gravité du skieur sur le ski} .
  \item Analyse des effets de \textbf{fatigue}, de \textbf{non-linéarité} et de la \textbf{variation des conditions de neige}.
\end{itemize}

\subsection{Optimisation et conception du ski selon la flexibilité}

\begin{itemize}
  \item Choix de \textbf{profils de raideur} adaptés. (Exemple : ski plus rigide au centre et plus souple aux extrémités.)
  \item Étude de l’influence du \textbf{profil en plan}, de la \textbf{section transversale} et des \textbf{matériaux}.
  \item Intégration des \textbf{contraintes de fabrication}, du \textbf{poids global} et du \textbf{comportement dynamique global}.
  \item \textbf{Objectif final} sur l'impact du \textbf{choix du ski }pour la pratique (loisirs, compétition)
\end{itemize}

\section{\textbf{Formules importantes de Flexion}}

\begin{itemize}
    \item Flèche à l'extrémité libre

    \[
\delta_{\text{max}} \;=\; \frac{F\,L^{3}}{3\,E\,I}
\]

où :
\begin{itemize}
  \item $\delta_{\text{max}}$ est la flèche maximale à l'extrémité libre de la poutre,
  \item $F$ est la force appliquée à l’extrémité libre,
  \item $L$ est la longueur de la poutre,
  \item $E$ est le module d’Young du matériau,
  \item $I$ est le moment quadratique de la section droite de la poutre.
\end{itemize}

\begin{itemise}
    \item Relation entre le moment de flexion et l'effort tranchant

    \[
\dv{M_{fz}(x)}{x} = T_y(x)
\]

où :
\begin{itemize}
  \item $M_{fz}(x)$ est le \textbf{moment de flexion} autour de l’axe $z$,
  \item $T_y(x)$ est l’\textbf{effort tranchant} selon l’axe $y$.
\end{itemize}

\noindent
Cette relation traduit l’\textbf{équilibre interne des efforts} dans une poutre fléchie :

\begin{itemise}
    \item Equation de la déformée pour une poutre encastrée

% Déformée d'une poutre encastrée (charge ponctuelle F à l'extrémité libre)
\[
y(x) \;=\; \frac{1}{E I(g,z)}
\Bigl(
-\frac{F}{6}\,(L-x)^{3}
-\frac{F L^{2}}{2}\,x
+\frac{F L^{3}}{6}
\Bigr)
\]

% Forme équivalente (classique) :
\[
y(x) \;=\; -\,\frac{F}{6\,E I(g,z)}\,x^{2}\bigl(3L - x\bigr)
\]

% Rappel des conditions aux limites (encastrement en x=0)
\[
y(0)=0, \qquad y'(0)=0
\]

Elle est déterminée à partir de 
  \[
y''(x) \;=\; \frac{F\,L^{3}}{3\,E\,I}
\]

\section{\textbf{Références bibliographiques}}
\noindent
\\
https://soothski.com/proprietes-skis-expliquees/2021/ski-longueur-sidecut-rayon-rockers-flexion-torsion-complique/?lang=fr \\\\\\
https://www.glissattitude.com/blog/ski-alpin-et-racing-11/souplesse-et-rigidite-des-skis-4737 \\\\\\
https://theses.insa-lyon.fr/publication/1986ISAL0006/these.pdf

\end{document}
