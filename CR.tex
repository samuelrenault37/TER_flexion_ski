\documentclass[12pt, a4paper]{article}
\usepackage[utf8]{inputenc}
\usepackage[T1]{fontenc}
\usepackage[french]{babel}
\usepackage{geometry}
\geometry{left=2.5cm, right=2.5cm, top=3cm, bottom=2.5cm, headsep=40pt}
\usepackage{amsmath, amssymb}
\usepackage{graphicx}
\usepackage{siunitx}
\usepackage{textgreek}
\usepackage{booktabs}
\usepackage{array}
\usepackage{float}
\usepackage{caption}
\usepackage{fancyhdr}
\usepackage{lastpage}
\usepackage{subcaption}
\usepackage{physics} % pour \dv et cie

\pagestyle{fancy}
\fancyhf{}
\fancyhead[L]{\includegraphics[height=1cm]{logo_em.jpg}}
\fancyhead[R]{MM1 Groupe 3}
\fancyfoot[C]{\thepage\ / \pageref{LastPage}}

\begin{document}

\begin{titlepage}
    \centering
    \includegraphics[width=0.5\textwidth]{logo_em.jpg}\\[5cm]
    
    {\LARGE \textbf{Travail d'Etude et de Recherche}}\\[0.3cm]
    {\large \textbf{Etude de la flexion sur des skis}}\\[1.5cm]
    {\large \textbf{Rapport}}\\[0.5cm]
    {\large \textbf{MM Groupe 1}}\\[7cm]
    
\\
    RUCHAUD--LANDRAUD Noah - SECK Seydou \\ RENAULT Samuel - POTAGNIK Martin\\[0.5cm]
    
    {\large \textbf{Tuteurs :} CATY Olivier - DURUFLE Marc}\\[2cm]
    
    {\large \textbf{Année 2025-2026}}
\end{titlepage}

% \maketitle  % (inutile avec la titlepage personnalisée)
\\\\
\begin{figure}[h!]
    \centering
    \includegraphics[width=0.32\textwidth]{IMG_2654.png}
    \includegraphics[width=0.32\textwidth]{IMG_2655.png}
    \includegraphics[width=0.32\textwidth]{IMG_2656.png}
    \caption{Récapitulatif de la séance 1 : (a) Ojectifs semestriels, (b) Carte mentale du projet, (c) Avancée travail hebdomadaire}
    \label{fig:flexion_ski}
\end{figure}

\section{\textbf{Idée plan de travail et d'avancée}}

\subsection{Modélisation mécanique du ski comme poutre (Théorie)}

\begin{itemize}
  \item Traitement du ski comme une \textbf{poutre} (linéaire ou non), avec une raideur en flexion $EI(x)$ variable selon la position $x$.
  \item Étude du \textbf{couplage torsion / flexion} : introduction du terme de raideur $GJ$ pour la torsion, pertinent si le ski est soumis à un moment de torsion en plus de la flexion.
  \item Analyse des \textbf{hypothèses de base} : petites déformations, comportement linéaire, homogénéité des matériaux.
  \item Discussion sur les \textbf{limites du modèle} : effets non-linéaires, anisotropie, hétérogénéité des couches.
\end{itemize}

\subsection{Mesure expérimentale de la flexion et de la rigidité (Réalité)}

\begin{itemize}
  \item Réalisation d'essais en \textbf{flexion 3 points, 4 points} ou avec appuis multiples pour déterminer le profil de raideur du ski. Idée de base avec la réalisation de l'exercice 1 du TD de RDM au préalable numériquement.
  \item Application de \textbf{charges connues} et mesure de la \textbf{déflexion} correspondante.
  \item Utilisation de \textbf{capteurs} (déplacement, jauges de contrainte, LVDT, capteurs de déformation) placés le long du ski. A voir si le matériel et si des capteurs sont disponibles au CRMI
  \item Comparaison des \textbf{données expérimentales} calculées avec les \textbf{modèles analytiques et numériques} qu'on trouverait en bibliographie sur Internet.
\end{itemize}

\subsection{Effet de la liaison ski-chaussure}

\begin{itemize}
  \item Étude de la manière dont la \textbf{fixation} et la \textbf{rigidité de la chaussure} influencent la distribution des efforts dans le ski.
  \item Analyse du \textbf{transfert des moments de flexion} de la chaussure vers le ski.
  \item Étude de l'\textbf{interaction du système complet} (skieur + chaussure + ski).
  \item   Etude en cas de \textbf{variations du sol} (bosses)
\end{itemize}

\subsection{Comportement en conditions réelles (dynamique)}

\begin{itemize}
  \item Mesure de la \textbf{déflexion du ski en mouvement} pendant les virages, à l’aide de capteurs embarqués (semble beaucoup plus difficile à réaliser expérimentalement).
  \item Étude de l’influence de la \textbf{vitesse}, de l’\textbf{angle d’attaque} et de la \textbf{charge dynamique}(mouvements du skieur, vibration du ski, changement de contact avec la neige).
  \item Analyse de l'importance de la position de la projection du \textbf{centre de gravité du skieur sur le ski} .
  \item Analyse des effets de \textbf{fatigue}, de \textbf{non-linéarité} et de la \textbf{variation des conditions de neige}.
\end{itemize}

\subsection{Optimisation et conception du ski selon la flexibilité}

\begin{itemize}
  \item Choix de \textbf{profils de raideur} adaptés. (Exemple : ski plus rigide au centre et plus souple aux extrémités.)
  \item Étude de l’influence du \textbf{profil en plan}, de la \textbf{section transversale} et des \textbf{matériaux}.
  \item Intégration des \textbf{contraintes de fabrication}, du \textbf{poids global} et du \textbf{comportement dynamique global}.
  \item \textbf{Objectif final} sur l'impact du \textbf{choix du ski }pour la pratique (loisirs, compétition)
\end{itemize}

\section{\textbf{Formules importantes de Flexion}}

\begin{itemize}
    \item Flèche à l'extrémité libre

    \[
\delta_{\text{max}} \;=\; \frac{F\,L^{3}}{3\,E\,I}
\]

où :
\begin{itemize}
  \item $\delta_{\text{max}}$ est la flèche maximale à l'extrémité libre de la poutre,
  \item $F$ est la force appliquée à l’extrémité libre,
  \item $L$ est la longueur de la poutre,
  \item $E$ est le module d’Young du matériau,
  \item $I$ est le moment quadratique de la section droite de la poutre.
\end{itemize}

    \item Relation entre le moment de flexion et l'effort tranchant

    \[
\dv{M_{fz}(x)}{x} = T_y(x)
\]

où :
\begin{itemize}
  \item $M_{fz}(x)$ est le \textbf{moment de flexion} autour de l’axe $z$,
  \item $T_y(x)$ est l’\textbf{effort tranchant} selon l’axe $y$.
\end{itemize}

\noindent
Cette relation traduit l’\textbf{équilibre interne des efforts} dans une poutre fléchie :

    \item Equation de la déformée pour une poutre encastrée

% Déformée d'une poutre encastrée (charge ponctuelle F à l'extrémité libre)
\[
y(x) \;=\; \frac{1}{E I(g,z)}
\Bigl(
-\frac{F}{6}\,(L-x)^{3}
-\frac{F L^{2}}{2}\,x
+\frac{F L^{3}}{6}
\Bigr)
\]

% Forme équivalente (classique) :
\[
y(x) \;=\; -\,\frac{F}{6\,E I(g,z)}\,x^{2}\bigl(3L - x\bigr)
\]

% Rappel des conditions aux limites (encastrement en x=0)
\[
y(0)=0, \qquad y'(0)=0
\]

Elle est déterminée à partir de 
  \[
y''(x) \;=\; \frac{F\,L^{3}}{3\,E\,I}
\]
\end{itemize}

\section{\textbf{Methodes numériques}}

L’objectif de cette partie est de modéliser la flexion du ski à l’aide d’outils numériques, afin de relier les équations théoriques de la poutre à des solutions approchées calculables.  
Pour cela, nous utiliserons la \textbf{méthode des différences finies}, qui permet de discrétiser les équations différentielles régissant la flexion et d’obtenir un système linéaire reliant les grandeurs mécaniques du ski.  
Ce travail constitue la première étape de modélisation numérique réalisée au semestre~5, avant d’exploiter, au semestre~6, les différents \textbf{solveurs numériques} qui permettront de résoudre efficacement les systèmes obtenus.

\subsection{Différences finies}

La méthode des différences finies est une méthode numérique permettant d'approcher
les solutions d'équations différentielles à l'aide d'équations algébriques.
L'idée principale est de remplacer les dérivées par des \textit{différences}
calculées à partir de valeurs de la fonction sur un ensemble discret de points.

\subsubsection{Principe général}

Soit une fonction $y(x)$ définie sur un intervalle $[a,b]$.
On choisit un nombre $N$ de points de discrétisation appelés \textit{nœuds} :
\[
x_0 = a,\quad x_1 = a + h,\quad x_2 = a + 2h,\quad \dots,\quad x_N = b,
\]
où $h = \dfrac{b - a}{N}$ est le pas du maillage.

\subsubsection{Forme de Newton (différences divisées)}

Soient des n\oe{}uds distincts $x_0,\dots,x_n$ et la notation de \textbf{différences divisées} :
\[
f[x_i] = f(x_i),\qquad
f[x_i,\dots,x_{i+k}] = \frac{f[x_{i+1},\dots,x_{i+k}] - f[x_i,\dots,x_{i+k-1}]}{x_{i+k}-x_i}.
\]

On définit la \textbf{base de Newton} :
\[
N_0(x)=1,\qquad
N_k(x)=\prod_{j=0}^{k-1}(x-x_j)\quad (k\ge 1).
\]

Ainsi on obtient la \textbf{forme finale du polynôme d'interpolation de Newton} :
\[
\boxed{
P_n(x)= f[x_0] + \sum_{k=1}^{n} f[x_0,\dots,x_k]\; N_k(x)
= \sum_{k=0}^{n} c_k\,N_k(x)
}
\]
avec 
\[
c_0=f[x_0],\quad c_1=f[x_0,x_1],\quad \dots,\quad c_n=f[x_0,\dots,x_n].
\]

\subsubsection{Écriture matricielle au niveau des n\oe{}uds.}
En évaluant $P_n$ aux n\oe{}uds $x_i$, on obtient le système :
\[
\underbrace{\begin{pmatrix}
N_0(x_0) & 0            & \cdots & 0 \\
N_0(x_1) & N_1(x_1)     & \cdots & 0 \\
\vdots   & \vdots       & \ddots & \vdots \\
N_0(x_n) & N_1(x_n)     & \cdots & N_n(x_n)
\end{pmatrix}}_{\displaystyle A}
\underbrace{\begin{pmatrix} c_0 \\ c_1 \\ \vdots \\ c_n \end{pmatrix}}_{\displaystyle \mathbf{c}}
=
\underbrace{\begin{pmatrix} f(x_0) \\ f(x_1) \\ \vdots \\ f(x_n) \end{pmatrix}}_{\displaystyle \mathbf{f}}
\quad\Longleftrightarrow\quad
A\,\mathbf{c}=\mathbf{f}.
\]

\noindent
Ici $A$ est \textbf{triangulaire inférieure} (dite matrice de Newton) car $N_k(x_i)=0$ pour $i<k$.  
Elle s’inverse facilement (tous les $x_i$ distincts $\Rightarrow \det(A)\neq 0$).

\subsubsection{Avantages et limites}

\paragraph{Avantages :}
\begin{itemize}
  \item Méthode simple à mettre en œuvre numériquement ;
  \item Bonne précision pour des maillages fins ($h$ petit) ;
  \item Convient bien aux géométries régulières (segments, carrés, cubes).
\end{itemize}

\paragraph{Limites :}
\begin{itemize}
  \item Moins adaptée aux géométries complexes ;
  \item La précision dépend fortement du pas $h$ ;
  \item Les conditions aux limites doivent être imposées avec soin.
\end{itemize}

\subsection{Solveurs numériques}

Les méthodes de différences finies conduisent à un système linéaire de la forme :
\[
A \, y = b,
\]
où $A$ est la matrice d'interaction entre les points de discrétisation, $y$ le vecteur des déformations, et $b$ le vecteur du chargement.

Pour résoudre ce système, on utilise un \textbf{solveur numérique}. Il existe deux grandes familles :
\begin{itemize}
\item les \textbf{solveurs directs}, comme l’élimination de Gauss ou la décomposition LU, qui fournissent une solution exacte après un nombre fini d’opérations;
  \item les \textbf{solveurs itératifs}, comme les méthodes de Jacobi, Gauss–Seidel ou Gradient Conjugué, qui construisent la solution progressivement jusqu’à convergence.
\end{itemize}

\section{\textbf{Références bibliographiques}}
\noindent
\\
https://soothski.com/proprietes-skis-expliquees/2021/ski-longueur-sidecut-rayon-rockers-flexion-torsion-complique/?lang=fr \\\\\\
https://www.glissattitude.com/blog/ski-alpin-et-racing-11/souplesse-et-rigidite-des-skis-4737 \\\\\\
https://theses.insa-lyon.fr/publication/1986ISAL0006/these.pdf

\end{document}
